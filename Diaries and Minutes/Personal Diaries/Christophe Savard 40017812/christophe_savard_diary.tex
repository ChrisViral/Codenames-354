\documentclass[letterpaper,10pt]{article}

\usepackage{microtype}

\topmargin=0cm
\oddsidemargin=0cm
\textheight=22.0cm
\textwidth=16cm
\parindent=0cm
\parskip=0.15cm
\topskip=0truecm
\raggedbottom
\abovedisplayskip=3mm
\belowdisplayskip=3mm
\abovedisplayshortskip=0mm
\belowdisplayshortskip=2mm
\normalbaselineskip=12pt
\normalbaselines

\begin{document}

\title{\Large{\textbf{Personal Diary - Iteration 1}}}
\author{Christophe Savard - 40017812}
\date{January 7\textsuperscript{th} to February 10\textsuperscript{th}}
\maketitle
\pagenumbering{gobble}
\clearpage

\tableofcontents
\pagenumbering{arabic}
\setcounter{page}{2}
\clearpage


\section{Week 1 (07/01-13/01)}

\subsection{Attendance}
Due to a doctor's appointment Thursday evening and the lack of Lab times on the first week of class, I was unable to attend this week's lecture and could not contact my other team members.

\subsection{Meeting summary}
Instead of procrastinating away, I decided to read the course documentation and information to get a head start.

\subsection{To-do list}
\begin{itemize}
\item Read the course outline
\item Read the project description for all three outlines
\item Familiarize myself with any tools used if not already familiar
\end{itemize}

\subsection{Work summary}
Read through the whole project description as mentioned. It's uh, interesting. At least I'm familiar with the game Codenames. Java is however a very strange choice of language to implement it in, and it seems a bit \textit{too} design pattern heavy. Class outline revealed of little interest. Still haven't managed to contact my team, but the teams seem to already be shifting.

\pagebreak


\section{Week 2 (14/01-20/01)}

\subsection{Attendance}
Christophe, Mottel, Shereece, Daniel, Michael, Rezza, Benjamin, Steven, Ashesh, Saad (missing one anonymous member)

\noindent Lab meeting: 21:30 to 23:00

\subsection{Summary}
After multiple team changes from the class drop-outs, I found myself in Mottel's team, a personal friend of mine. He promptly invited me into the Discord server he made for his team which he has already met, and I attended the (very late) lab and met everyone. The main topic of discussion was which technologies were to be used. I managed to convince the team to use a public organization on GitHub for source management, as well as using GitKraken Pro to interface with GitHub easily, since everyone has access to it for free using GitHub's education package. I also easily got the team on board to use IntelliJ instead of Eclipse as an IDE, as it is much more powerful and versatile, and available for free for all Concordia students.

The other agreed on tools were Maven for package management, JUnit for unit testing, Swing for UI, and SQLite for database interaction.
\bigbreak
\noindent The teams were also separated as follows:

\noindent \textbf{Coders:} Me, Ben, Steven

\noindent \textbf{Organizers:} Mottel, Dan

\noindent \textbf{Documenters:} Rezza, Shereece, Saad

\noindent \textbf{QA:} Mike, Ashesh
\bigbreak
\noindent The team also tasked me with handling the GitHub business, which I'm happy to do as I know Git extensively and feel much safer knowing I'll be managing the repository.

\subsection{To-do list}
\begin{itemize}
\item Install all tools on both my home machine and laptop
\item Familiarize myself with the game and think of how I would personally implement it
\item Learn how Swing works
\item Setup the GitHub organization and the repository as well as the base project
\item Learn the design patterns that will need to be used for the project, specifically the MVC pattern
\end{itemize}

\subsection{Work summary}
I created the organization and repository, and sent everyone invites to join the organization, then separated people into specific teams. I then setup the Java project in IntelliJ with Maven then added the SQLite and JUnit dependencies. Once I got the project working, I pushed everything to git and made sure to protect the master branch.

\pagebreak


\section{Week 3 (21/01-27/01)}

\subsection{Attendance}
Chris, Mottel, Shereece, Dan, Mike, Rezza, Ben, Steven, Ashesh, Bilal

\noindent Lab meeting (via Discord): 21:30 to 23:00

\subsection{Meeting summary}
Met this week via Discord, due to inclement weather and me living rather far away. The meeting felt a bit more like a progress report and to restructure a bit everyone's expectations and workloads. Saad dropped the class and the missing member, Bilal, was presented and added to the Organizers.

\subsection{To-do list}
\begin{itemize}
\item Start working on the structure of the code
\item Meet with the other members of the code team to discuss how to proceed and split the task
\end{itemize}

\subsection{Work summary}
Haven't been able to put much work this week due to different conflicting assignments and personal events taking my time. Talked a bit with the coders, and we decided to instead switch to using JavaFX for the UI to take advantage of the SceneBuilder.

\pagebreak


\section{Week 4 (28/01-03/02)}

\subsection{Attendance}
Chris, Mottel, Shereece, Dan, Mike, Rezza, Ben, Steven, Ashesh, Bilal

\noindent Lab meeting: 21:30 to 23:00

\subsection{Meeting summary}
The lab was a "practice" demo where we ran the code and had the TA give us pointers. He noted that all files should have a header comment indicating who worked on them, etc. After the demo, we did a progress report and everyone indicated what they had done and what they were planning to do. I helped Ashesh around with GitKraken and with the idea of Unit tests. Steven and Ben have started writing some boilerplate code for the base of the Model

\subsection{To-do list}
\begin{itemize}
\item Start writing the structure of the MVC
\item Setup the View through FXML
\item Get as much working as possible
\end{itemize}

\subsection{Work summary}
I worked heavily on the code. Mottel provided a basic FXML file and CSS for styling, which I integrated in the project and got running. I linked everything together and made started assigning words from the database to the cards on the screen, then proceeded to assigning correct colours, making sure the amount of cards was respected, and enabled click interaction to test flip the cards. I also made sure that the FXML and CSS files were part of the resources of the project, then created a JAR build process, and tested it to make sure it worked.

I also went through the Git history to see who worked on which files and created header comments for each file with the creation date, creator, and contributors.

\pagebreak


\section{Week 5 (04/02-10/02)}

\subsection{Attendance}
Chris, Mottel, Shereece, Dan, Mike, Rezza, Ben, Steven, Ashesh, Bilal

\noindent Lab meeting (via Discord): 21:30 to 22:30

\subsection{Meeting summary}
After an extreme amount of confusion, turns out we do \textit{not} have a graded demo this week, contrary to the very conflicting information provided by the TA and teacher, and the due date for the submission being this Sunday. I stayed home due to the crazy weather and icy rain, and dialed in to the meeting via Discord once more. Another progress report, and seeing what was missing for Sunday's submission.

\subsection{To-do list}
\begin{itemize}
\item Finalize the game's code
\item Create the Commander pattern
\item Link together the AI to the board
\item Setup the observer pattern connection from the Model to the Controller
\end{itemize}

\subsection{Work summary}
Successfully implemented the Observer pattern via a series of Events, then linked all the information from the Game to the fields in the View through the Controller. I deactivated the click interaction, then created the commander pattern for logging and to undo card flips. After further review, it was realized that the Undo/Redo action is actually not necessary for this Iteration, so it was hidden away, but not fully scrapped, so it can serve as a basis during the next iteration. Ben and Steven worked on the AI and game loop. After all was done, I linked everything together to make sure the interaction from the View was smooth and the game played on nicely.

Once the code was complete, I helped Mike a bit with unit testing, then did a massive cleanup of the codebase, simplifying the code when possible and making sure to comment throughout and structure the class layout, and also added missing Javadocs. I built the final JAR and successfully tested it. Booted my Linux VM to compress the source into the required tarball, then submitted.

\pagebreak

\end{document}
