\documentclass[12pt]{article}

\pagestyle{empty}
\setcounter{secnumdepth}{2}

\topmargin=0cm
\oddsidemargin=0cm
\textheight=22.0cm
\textwidth=16cm
\parindent=0cm
\parskip=0.15cm
\topskip=0truecm
\raggedbottom
\abovedisplayskip=3mm
\belowdisplayskip=3mm
\abovedisplayshortskip=0mm
\belowdisplayshortskip=2mm
\normalbaselineskip=12pt
\normalbaselines

\begin{document}

\vspace*{0.5in}
\centerline{\bf\Large Personal Journal}
\vspace*{0.5in}
\centerline{\bf\Large Team PJ-b}
\vspace*{0.5in}
\centerline{\bf\Large Project Title: Codenames Game }
\vspace*{0.25in}
\centerline{\bf\Large Iteration 1}


\vspace*{2in}
\begin{table}[htbp]
\caption{Team}
\begin{center}
\begin{tabular}{|l | c|}
\hline
Contributor: & ID Number \\
\hline\hline
Shereece A.A. Victor  & 40105094  \\
\hline
\end{tabular}
\end{center}
\end{table}

\clearpage
\section{Introduction}

\subsection{Project Description}
According to my understanding thus far, Codenames Game, is a spy themed card game, designed for play with four or more persons. It involves, two Spymasters, and Operatives. The players are divided into two teams, a red team and a blue team. 

The game is played with a mixture of red and blue cards placed faced down on the table in a 5x5 grid. The contents and placement of each card are only known to each spymaster. The cards each contain a word which is visible to all, and either a red or blue spy, uncolored civilian or the assassin. 

Throughout the game each team’s spymaster is tasked with provided clues that indicated how many and the location of their color spies on the board. The first team to figure out the location of all their spies wins. If either team reveals the assassin, they lose and the game ends. 

The goal of my team’s project to recreate this game in a computer program in three iterations. The first, gives the user minimum input, and the last allows the a user to participate in the game. 

\subsection{Team Members}

PJ-b Initial Team Members:
\begin{itemize}
	\item BT - Benjamin Thérien 
	\item DT - Daniel Thibault-Shea
	\item MZ - Mordechai Zirkind 
	\item RZ - Rezza Zairan
	\item SV - Shereece Victor
	\item SZ - Steven Zanga 
\end{itemize}


PJ-b Final Members
\begin{itemize}
	\item AP - Ashesh Patel 
	\item BR - Bilal Rana
	\item BT - Benjamin Therien
	\item CS - Christophe Savard
	\item DS – Daniel Thibault-Shea
	\item MW - Michael Wilgus
	\item MZ - Mordechai Zirkind
	\item SV - Shereece Angell Agatha Victor
	\item SZ - Steven Zanga
\end{itemize}
 

\pagebreak
\section{Team Meeting 1a }

\subsection{Administrative}
Wednesday, 16th January, 2019 | 8:15pm – 9:15pm (approx.) | H Building 7th Floor Common Area 
Attended by: 
BT - Benjamin Thérien 
DT - Daniel Thibault-Shea 
MZ - Mordechai Zirkind 
RZ - Rezza Zairan 
SV - Shereece Victor 
SZ - Steven Zanga


\subsection{Summary}
-	To begin the meeting, we played the card game, Codenames, to familiarize ourselves with the gameplay and rules. 
-	After the game I understood finally understood the game. 
-	I presented by idea of using a 5x5 matrix data structure to represent the game board and assigning each of the matrix’s contents to a one-dimensional array for ease of use. Then using a random function to assign the cards to each space. 
-	MZ initiated the discussion of how we would implement the game and spoke about logical game design using objects.
-	SZ and BT retaught the MVC model, because we were still unclear of it. 
-	We all participated in a discussion about where the logic of the system would be contained, if it would be in the controller or the model. 
-	We discovered that our group had been expanded and thus we went to the lab to meet them. 



\pagebreak
\section{Team Meeting 1b }

\subsection{Administrative}
Wednesday, 16th January 2019 | 9:30 pm – 10:40pm (approx.) | H 903 Lab 
Attendees: AP – Ashesh Patel*
SP – Saad Patel*
BT - Benjamin Thérien
CS – Christophe Savard* 
DT - Daniel Thibault-Shea
MZ - Mordechai Zirkind 
MW – Micheal Wilgus* 
MZ - Mottel Zirkind
RZ - Rezza Zairan 
SV - Shereece Victor 
SZ - Steven Zanga



\subsection{Summary}
-	We met the new members of our team (labelled with an ‘*’).
-	We met the tutor, he gave us guidance on what is expected.
-	The different roles were discussed: Coders, Documenters, Organizers, Quality Assurance.
-	We chose our roles for this iteration, I am a documenter, because I didn’t do any Java programming prior to this course and needed time to learn and adapt. 
-	It was declared that everyone will be involved in coding and testing. 
-	We discussed the skills needed: knowledge of Java, Unit Testing, GitHub etc.
-	I think I am the least knowledgeable and experienced in this group. 
-	The software, websites and accounts we will be using to go the project. 
-	The MVC model was recapped. 
-	We discussed the database that may be needed, and maybe storing words, their hints and even pre-generated game boards in them. 
-	We discussed, what game statistics would be displayed, how intelligent the computer players should be, the game cycle, and tasks to do by next week. 

 



\pagebreak

\section{Team Meeting 2 }

\subsection{Administrative}
Wednesday, 23rd January 2019 | 9:30 pm – 10:40pm (approx.) | Capstone Project room 
Attendees: 
Via Discord: 
BR - Bilal Rana
CS – Christophe Savard 
SZ - Steven Zanga
Present: 
AP – Ashesh Patel
BT - Benjamin Thérien
DT - Daniel Thibault-Shea
MW – Micheal Wilgus
MZ - Mordechai Zirkind  
RZ - Rezza Zairan 
SV - Shereece Victor 




\subsection{Summary}
-	We discussed what we each want out of the course, I want to not be too stressed. 
-	We recapped what we each did that week. 
-	I had written and posted the meeting minutes and downloaded a Latex editor. 
-	Documenters were assigned the task or drawing UML diagrams.
-	We realised a flaw in our work as a team, we were trying to write code before establishing the scope and requirements of the project. 
-	Learned about the code written so far, how they fit into the MVC model and got an idea of what objects will be used and how they come together to form the game 
-	Finally, I suggested that team mates share their progress with the group to improve synergy. 




\pagebreak

\section{Personal work }

\subsection{Administrative}
Tuesday, 29rd January 2019 | 3:00 pm – 4:05 pm | -



\subsection{Summary}
-	Reviewed code posted on GitHub
-	Downloaded Latex Document template and making edits to see how it works   




\pagebreak
\section{Mini Team Meeting }

\subsection{Administrative}
Wednesday, 30th January 2019 | 3:00 pm – 6:00 pm | LB Sandbox
Attendees: 

MZ - Mordechai Zirkind  
RZ - Rezza Zairan 
SV - Shereece Victor 



\subsection{Summary}
-	Discussed the gameplay with MZ 
-	Discussed the format of the use cases and their format with RZ 
-	Researched use cases and examples of use cases 
-	Started writing main use cases in Word Document 



\pagebreak
\section{Mini Team Meeting }

\subsection{Administrative}
Wednesday, 30th January 2019 |7:00 pm – 9:25 pm | H Building 6th Floor Common Area 
Attendees: 

MW – Micheal Wilgus
MZ - Mordechai Zirkind  
RZ - Rezza Zairan 
SV - Shereece Victor 




\subsection{Summary}
-	Transferred Use Cases to Latex Document 
-	Spent hours trying to debug syntax errors
-	MW found the solution at the end 
-	Discussed UML and Use Case diagrams 
-	RZ suggested a good software for drawing UML diagrams 



\pagebreak
\section{Team Meeting 3 }

\subsection{Administrative}
Wednesday, 30th January 2019 |9:30 pm – 10:30 pm | H 903 Lab
Attendees: 

AP - Ashesh Patel
BR - Bilal Rana
CS - Christophe Savard
DS – Daniel Thibault-Shea
BT - Benjamin Therien
SV - Shereece Angell Agatha Victor
MW - Michael Wilgus
RZ - Rezza-Zairan Zaharin
SZ – Steven Zanga
MZ - Mordechai Zirkind





\subsection{Summary}
-	Discussed Progress with Tutor 
-	I created a list of tasks to be completed by next week based on the requirements mentioned by the tutor:

-	Clarified again, the work to be done for the first iteration 




\pagebreak


\end{document}
