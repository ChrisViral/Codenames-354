\documentclass[10pt, a4paper]{article}

%import packages
\usepackage[pagestyles]{titlesec}
\usepackage{titlesec}
\usepackage{enumerate}

%package configurations
\pagestyle{empty}	
\setcounter{tocdepth}{4}
\renewcommand*\contentsname{\centering Table of Contents}
\titleformat*{\section}{\center\Large\bfseries}

%configures page number placement
\newpagestyle{main}{\setfoot{}{}{\thepage}}
\pagestyle{main}
\assignpagestyle{\chapter}{main}

%stuff from prof that handles formatting
\topmargin=0cm
\oddsidemargin=0cm
\textheight=22.0cm
\textwidth=16cm
\parindent=0cm
\parskip=0.15cm
\topskip=0truecm
\raggedbottom
\abovedisplayskip=3mm
\belowdisplayskip=3mm
\abovedisplayshortskip=0mm
\belowdisplayshortskip=2mm
\normalbaselineskip=12pt
\normalbaselines


\begin{document}
		
\hrulefill
\begin{flushright}
\textbf{Team PB-J}
\end{flushright}
\hrulefill

\vspace*{0.5in}
\centerline{\bf\Large CODENAMES GAME}

%team members table
\vspace*{1.5in}
\begin{table}[htbp]
\begin{center}
	Team Members\\
\end{center}
\begin{center}
	\begin{tabular}{| c | c |}
	\hline
	Name & ID Number \\
	\hline\hline
	Ashesh Patel & ??? \\
	Christophe Savard & 40017812\\
	Benjamin Th\'erien & 40034572\\
	Daniel Thibault-Shea & 40073133\\
	Shereece Victor & 40105094\\
	Michael Wilgus & 29206388 \\
	Rezza-Zairan Zaharin & 40003377 \\
	Steven Zanga & 40000797\\
	Mottel Zirkind & 27206151\\
	\hline
	\end{tabular}
\end{center}
\end{table}

\newpage
%\tableofcontents
\newpage


\section{Project Analysis and Development Plan}

	\subsection{Introduction}
	
	 The purpose of this document is to detail the high-level requirements and features of the Codenames Game developed by team PB-J. The Codenames Game is a game of 2 to 8 human and AI players. This project is an adaptation of the Codenames Game designed by Vlada Chv\'atil and published by Czech games.\\
	 \\
	 The specifics of how the Codenames Game fulfills these needs will be detailed in the use cases which more will be detailed in the upcoming design phase\\
	
	\subsection{Purpose}
	
	This document will describe the specifications entailed by the development of Codenames in compliance with the requirements of COMP 354. It will outline the high-level requirements encompassing user interfaces, product functions, user descriptions, assumptions and dependancies, constraints, specific requirements and an analysis model. The analysis model will hold use case diagrams, class diagrams, sequence diagrams and state transition diagrams.
	
	\subsection{Scope}
	
	This document only addresses the high level requirements of Codenames  that the design phase.  \\
	

	\subsection{Definitions and Abbreviations}
	
		\subsubsection{Definitions}
	
		\textbf{Board} \\
		\\
		The main playing area will be composed of a 5 by 5 grid of words. Each cell of the grid are representative of codenames of agents. Each cell in the grid will be defined in this document as a \textbf{Card}.\\
	
		\textbf{Card} \\
		\\
		The card will hold two values indicative of it's state: unrevealed
		and revealed. In it's unrevealed state, it will present a word taken from the game. In it's revealed state, it can either be a blue or red team's \textbf{Spy} card, a \textbf{Civillian} card or the \textbf{Assassin} card.\\
		
		\textbf{Assassin} \\
		The Assassin is a card type that when revealed, it would cause the team that revealed it to lose.\\
		
		\textbf{Spy}\\
		\\
		The Spy is a card type that belongs to either the red or blue team. Once all spy cards of a team are revealed, the team wins.
		
		\textbf{Civillian}\\
		\\
		The Civillian is a card type that occupies leftover space on the board that is not occupied by \textbf{Spy} cards or the \textbf{Assassin} card. When revealed, it only skips the revealing team's turn. \\
	
		\subsubsection{Abbreviations}
		
		To be updated. \\
	
	\subsection{Reference}
	
	To be updated.\\
	
	\subsection{Overview}
	
	The rest of this document outlines the problem description and the development plan. \\ 
	\\  
	The problem description will describe the game user's interfaces, product functions, user descriptions, assumptions and dependencies, constraints, specification requirements, and the analysis model. \\  
	
	\newpage
	
\section{Problem Description}

	\subsection{Project Purpose, Scope, and Objectives}
	
	% Update for each iteration
	The objective of this project is to simulate a multi-user tabletop game named Codenames game by by Vlada Chv\'atil and published by Czech games. The rules of the game will intimately follow Vlada Chv\'atil with slight variations to accomodate the digital conversion. This project will be a multi-user game of up to four players composed of currently only computer players.\\
	
		\subsubsection{User Interfaces}
	
			\paragraph{Game Board}
			
			\paragraph{Card}
			
			\paragraph{Winner Interface}
			
	\subsection{Product Functions}
	
	Every function below has to support system  functions, such as a click of a button or revealing images when necessary. The following functions will be a part of the Codenames game.\\
	
		\subsubsection{Introduction}
		
		To be updated.\\
		
		\subsubsection{Board}
		
		To be updated.\\
		
		\subsubsection{Game}
		
		There will be two teams of \textbf{red} and \textbf{blue}. Each team possesses a pair of players each playing either the role of Spymaster of Operative. Since each team follows the same path:\\
		
		\begin{itemize}
			\item [--] Team \textbf{Spymaster} reveals clue.
			\item [--] Turn passes to Team's \textbf{Operative}
			\item [--] Team \textbf{Operative} makes guesses based on clue given.
			\item [--] Turn passes to opposing team's \textbf{Spymaster}
		\end{itemize}
	
		\textbf{Input:}\\
		To be updated.\\
		\\
		\textbf{Action:}
		To be updated.\\
		\\
		\textbf{Output:}\\
		To be updated.\\
		\\
		\textbf{Validity Check:}\\
		Sequence of the players is to be followed according to the order set earlier established. When a team finishes their turn, the next team's turn becomes active.\\
		
		
	\subsection{User Description}
	
		\subsubsection{User Environment}
		
		\subsubsection{User Profiles}
		
	\subsection{Assumptions and Dependencies}
	
	\subsection{Constraints}
	
	Assuming that a majority of player's PCs will run by the Windows OS, the project must be written in this platform that supports GUI(Graphical User Interface). Thus, the decision has been made to use :
	\begin{itemize}
		\item[--] JAVA as the programming language.
		\item[--] SQLite as chosen for data storage.
		\item[--] J-Unit for unit testing.
		\item[--] Eclipse as the integrated development environment(IDE).
	\end{itemize}
		
	
	\subsection{Specific Requirements}
	
	\subsection{Analysis Models}
	
		\subsubsection{Use Case Diagrams}
		
		The following diagrams will help provide an overview of the functions in the game. They describe the action that a player can perform, as well as the interaction between some of the system functions, ... To be updated.\\
		
		\subsection{Use Cases?}
		
		\subsubsection{Use Case Details}
			\paragraph{Use Case 1: Start Game }
			
			\begin{center}
				\begin{tabular}{ |c|p{7cm}| } 
					\hline
					Description & The user commences the game  \\
					\hline 
					Actors & User \\
					\hline  
					Pre-Conditions & None \\
					\hline  
					Basic Path & 
					
					\begin{itemize}
						
						\item The user clicks "Start Game".
					\end{itemize}\\
					\hline
					Alternative Paths & None \\
					\hline
					Post-Conditions & \begin{itemize}
						\item The Board is initialized.
						\item The first Spymaster can reveal a clue. 
					\end{itemize} \\
					\hline 
					Related Use Cases & \\
					\hline 
					Used Use Cases & None \\
					\hline 
					Extending Use Cases & None \\
					\hline 
				\end{tabular}
			\end{center}
	
	
	\paragraph{Use Case 2: Reveal Clue}
		\begin{center}
		\begin{tabular}{ |c|p{7cm}| } 
			\hline
			Description & The Spymaster issues a clue \\ 
			\hline
			Actors & Spymaster \\
			\hline 
			Pre-Conditions & \begin{itemize}
				\item The Board is initialized.
				\item It's the Spymaster's turn to play  
			\end{itemize} \\
			\hline
			Basic Path & 
				\begin{enumerate}
					\item The word which comprises the clue is displayed 
					\item The number of cards related to the clue is revealed
					\item 3.	The system checks to see if the clue is valid 
				\end{enumerate} \\
			\hline 
			Alternative Paths & Alternative 1: \begin{itemize}
				\item If the clue is not valid, a card belonging to the opposing team is revealed. 
				\item The turn is passed to the opposing spymaster.
			\end{itemize}
	
			Alternative 2:
			\begin{itemize}
			\item	The clue is valid
			\item 	Game play continues 
			\end{itemize}
			 \\
			\hline 
			Post-Conditions & 
				\begin{itemize}
					\item A clue has been revealed
					\item The Spymaster’s turn has ended 
				\end{itemize} \\
			\hline 
			Related Use Cases & \\
			\hline 
			Used Use Cases & None \\
			\hline
			Extending Use Cases & None \\
			\hline
		\end{tabular}
	\end{center}
		
		\paragraph{Use Case 3: Card Reveal}
		\begin{center}
			\begin{tabular}{ |c|p{7cm}| } 
				\hline
				Description & The operative picks cards to be revealed\\ 
				\hline
				Actors & Operative  \\
				\hline 
				Pre-Conditions & \begin{itemize}	
					\item The Spymaster has given a valid clue and number of guesses 
					\item It is the operatives turn to play
				\end{itemize}	
				  \\
				\hline
				Basic Path & 
				\begin{enumerate}
					\item The operative picks a card on the board based on the clue
					\item The system reveals the contents of the card  
					\item If the card chosen belongs to the operative’s team. The operative’s reveal count increments.
					\item The operative gets to reveal another card 
				\end{enumerate} \\
				\hline 
				Alternative Paths & Alternative 1: \begin{itemize}
					\item If the operative has depleted their chances to guess, they cannot reveal another card.
				\end{itemize}
				
				Alternative 2:
				\begin{itemize}
					\item If the operative reveals the opposing team’s card; the opposing teams reveal count is incremented.
					\item The operative’s turn ends.  
				\end{itemize}
			
			Alternative 3:
			\begin{itemize}
				\item If the operative reveals a civilian card: 
				\item The operative’s turn ends   
			\end{itemize}
		
		Alternative 4:
		\begin{itemize}
			\item If the operative reveals the assassin, the game ends.
			\item The operative’s team loses  
		\end{itemize}
				\\
				\hline 
				Post-Conditions & 
				\begin{itemize}
					\item A clue has been revealed
					\item The Spymaster’s turn has ended 
				\end{itemize} \\
				\hline 
				Related Use Cases & \\
				\hline 
				Used Use Cases & None \\
				\hline
				Extending Use Cases & None \\
				\hline
			\end{tabular}
		\end{center}
			
			\paragraph{Use Case 4: End Game}
			\begin{center}
				\begin{tabular}{ |c|p{7cm}| } 
					\hline
					Description & The game is ended \\ 
					\hline
					Actors & System  \\
					\hline 
					Pre-Conditions & \begin{itemize}	
						\item The assassin card has been revealed 
						\item One of the teams has revealed all of their cards
					\end{itemize}	
					\\
					\hline
					Basic Path & 
					\begin{enumerate}
						\item The game board is hidden
						\item A results screen is displayed   
					\end{enumerate} \\
					\hline 
					Alternative Paths & 
					\\
					\hline 
					Post-Conditions & 
					\begin{itemize}
						\item The game is done
						
					\end{itemize} \\
					\hline 
					Related Use Cases & \\
					\hline 
					Used Use Cases & None \\
					\hline
					Extending Use Cases & None \\
					\hline
				\end{tabular}
			\end{center}
	
		\subsubsection{Class Diagrams}
		
			\paragraph{Full Class Diagram}
			
			\paragraph{Simplified View}
			
			\paragraph{Hierarchical View}
			
		\subsubsection{Sequence Diagram}
		
\section{Development Plan}

This section of the document contains the estimated cost and schedule for the project in the form of a phase plan and project schedule. To be updated. \\
	\subsection{Project Estimates}
	
	\subsection{Project Plan}
	
		\subsubsection{Phase plan}
		\subsubsection{Project Schedule}
		\subsubsection{Project Resourcing}
	
	
\end{document}
